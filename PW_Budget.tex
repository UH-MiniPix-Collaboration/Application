\section{Power and Weight Budget}
\label{sec:PW_Budget}

%We will copy the thing from the provisional application and update it with new values....
%\begin{table}[H}
%\centering
%\label{budget}
%	\begin{tabular}{|c|c|c|c|c|c|}\hline
%	\caption{Power and weight budget for SORA 2.0}
%	Component & Voltage (VDC) & Current (mA) & Duty Cycle (\%) & Power (mW) & Weight (g)\\ \hline
%	30 to 12 V DC/DC Converter & 30 &  & 100 & NA & NA \\ \hline
%	Polulu Driver & 30 &  & 100 & NA & NA \\ \hline
%	Polulu Driver & 30 &  & 100 & NA & NA \\ \hline
%	RESU & 12 & 65 & 100 & 780 & 9 \\ \hline
%	Pump 1 Heater & 12 &  & 10 & NA & NA \\ \hline
%	Pump 2 Heater & 12 &  & 10 & NA & NA \\ \hline
%	Pump 1 & 24 & 270 & 75 & 6480 & 1769.01 \\ \hline
%	Pump 2 & 24 & 270 & 75 & 6480 & 1769.01 \\ \hline
%	Solenoid 1 & 12 & 241.67 & 50 & 2900.04 & 5 \\ \hline
%	Solenoid 2 & 12 & 241.67 & 50 & 2900.04 & 5 \\ \hline
%	Servo Motor & 12 & 7 & 0.1 & 84 & 10 \\ \hline
%	Temperature Sensor 1 & 5 & 0.09 & 100 & 0.45 & 0.02 \\ \hline
%	Temperature Sensor 2 & 5 & 0.09 & 100 & 0.45 & 0.02 \\ \hline
%	Temperature Sensor 3 & 5 & 0.09 & 100 & 0.45 & 0.02 \\ \hline
%	Temperature Sensor 4 & 5 & 0.09 & 100 & 0.45 & 0.02 \\ \hline
%	Temperature Sensor 5 & 5 & 0.09 & 100 & 0.45 & 0.02 \\ \hline
%	Temperature Sensor 6 & 5 & 0.09 & 100 & 0.45 & 0.02 \\ \hline
%	Temperature Sensor 7 & 5 & 0.09 & 100 & 0.45 & 0.02 \\ \hline
%	Pressure \& Altitude Sensor 1 & 3.3 & 1.4 & 100 & 4.62 & 0.005 \\ \hline
%	Real Time Clock (RTC) & 3.3 & 0 & 100 & 0.00198 & 0.005 \\ \hline
%	GPS & 3.3 & 41 & 100 & 135.3 & 0.005 \\ \hline
%	Real Time Clock (RTC) & 3.3 & 0 & 100 & 0.00198 & 0.005 \\ \hline
%	Humidity Sensor & 3.3 & 0.5 & 100 & 1.65 & 0.0025 \\ \hline
%	BNO Multisensor & 3.3 & 0.1 & 100 & 0.33 & 0.005 \\ \hline
%	Total &  & 1138.97 &  & 22839.53 & 3992.24 \\ \hline
%	
%	\end{tabular}
%\end{table}

\begin{table}[H]
\centering
\caption{Power and weight budget for SORA 2.0} 
\label{tab:budget}
\bigskip
\begin{tabular}{|c|c|c|c|c|c|}
\hline
\multicolumn{1}{|c|}{\bfseries Component} & \multicolumn{1}{c|}{\bfseries Voltage (VDC)} &  \multicolumn{1}{c|}{\bfseries Current (mA)} & \multicolumn{1}{c|}{\bfseries Duty Cycle (\%)} & \multicolumn{1}{c|}{\bfseries Power (mW)} & \multicolumn{1}{c|}{\bfseries Weight (g)} \\
\hline
    30 to 12 V DC/DC Converter & 30 & 1500 & 100 & 45000 & 10 \\ \hline
	%Polulu Driver & 30 & 1000 & CAL & CAL & 1 \\ \hline
	%Polulu Driver & 30 & 1000 & CAL & CAL & 1 \\ \hline
	RESU and MiniPIX~\ref{tab:Sensors} & 12 & 290 & 100 & 3480 & 15 \\ \hline
	Pump 1 Heater with driver & 12 & 180 & 40 & 2160 & 2 \\ \hline
	Pump 2 Heater with driver & 12 & 180 & 40 & 2160 & 2 \\ \hline
	Pump 1 w/ Solenoid & 24 & 670 & 80 & 16080 & 1800 \\ \hline
	Pump 2 w/ Solenoid & 24 & 670 & 80 & 16080 & 1800 \\ \hline
	%Solenoid 1 & 12 & 241.67 & 50 & 2900.04 & 5 \\ \hline
	%Solenoid 2 & 12 & 241.67 & 50 & 2900.04 & 5 \\ \hline
	%Servo Motor & 12 & 7 & 0.1 & 84 & 10 \\ \hline
	%Temperature Sensor 1 & 5 & 0.09 & 100 & 0.45 & 0.02 \\ \hline
	%Temperature Sensor 2 & 5 & 0.09 & 100 & 0.45 & 0.02 \\ \hline
	%Temperature Sensor 3 & 5 & 0.09 & 100 & 0.45 & 0.02 \\ \hline
	%MiniPIX & 5 & 0.09 & 100 & 0.45 & 0.02 \\ \hline
	%Pressure \& Altitude Sensor 1 & 3.3 & 1.4 & 100 & 4.62 & 0.005 \\ \hline
	%Real Time Clock (RTC) & 3.3 & 0 & 100 & 0.00198 & 0.005 \\ \hline
	%GPS & 3.3 & 41 & 100 & 135.3 & 0.005 \\ \hline
	%Real Time Clock (RTC) & 3.3 & 0 & 100 & 0.00198 & 0.005 \\ \hline
	%Humidity Sensor & 3.3 & 0.5 & 100 & 1.65 & 0.0025 \\ \hline
	%BNO 6055 & 3.3 & 0.1 & 100 & 0.33 & 0.005 \\ \hline
	Clean Box & N/A & N/A & N/A & N/A & 10000 \\ \hline
	Structure w/ bolts & N/A & N/A & N/A & N/A & 5000 \\ \hline
	Total & 30 & 2.1 (peak) & 100 & 45000 & 18764 \\ \hline
\end{tabular}
\medskip
\end{table}


\subsubsection{Powering It All Up}

In order to stay within the power constraints, a robust power supply will need to  handle all the components of the payload.  The power supply we will be using is the PPM-DC-ATX-P by WinSystems INC.  It offers the desired number of \SI{+5}{\volt} and \SI{+12}{\volt} outputs needed to power the payload's electronics.  This power supply could effectively take \SI{+30}{\volt} and step it down to two \SI{+12}{\volt} and two \SI{+5 }{\volt} outputs.  One of the \SI{+12}{\volt} outputs goes to the Arduino since it can step down to the appropriate voltages internally while the other goes to a PWM motor for the solenoid.  One of the \SI{+5 }{\volt} outputs powers two analog sensors that will be sent to HASP through the EDAC connection (more on that in the next sections).  The remaining \SI{+5 }{\volt} output is converted to a USB power cable for the RP3.  The power supply also has four ground outputs that will be used by each respective component. Table~\ref{tab:budget} shows all expected power consumption from subsystems and sensors for the entire payload.
