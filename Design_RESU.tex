\subsection{RESU Design}
\label{sec:RESU-Design}

For this mission, we will again use RESU (Real-Time Environmental Sensing Unit), our flight computer that can manage flight operations as well as send and receive serial uplink and downlink packets. RESU is composed of two components: a Raspberry Pi 3 (RP3) and an Arduino Mega (Arduino) which interfaced over serial USB.  The RP3 and Arduino both are hobby electronic computers, but the RP3 is geared towards recording and processing while the Arduino is more efficient at sensor integration.  The Arduino's primary purpose is to collect data from various sensors that monitor environmental conditions. It also can handle telemetry, periodically downlinking packets and accepting uplink commands received from the ground control. The RP3 records data from the Arduino's serial port and saves data frames every six seconds from the MiniPIX.  Table~\ref{tab:Sensors} lists the various sensors that will be utilized during flight.

\begin{table}[h!]
\centering
\caption{Table of sensors that compose RESU}
\label{tab:Sensors}
\bigskip
\begin{tabular}{|c|c|c|c|}
\hline
\multicolumn{1}{|c|}{\bfseries Sensor} & {\bfseries Quantity} & {\bfseries Platform} & {\bfseries Purpose} \\
\hline
    Temperature Sensors         	& 3 & Arduino  		& Record temperature measurements  \\ \hline
    Pressure        				& 1 & Arduino 		& Record pressure measurements \\ \hline
    Inertial Measurement Unit       					& 1 & Arduino 		& Record IMU Data in 9 degrees of freedom \\ \hline    
    Real Time Clock 				& 2 & Arduino/RP3 	& Record temperature compensated timestamps in CT \\\hline
    Humidity        				& 1 & Arduino 		& Record atmospheric humidity levels \\ \hline
    MiniPIX         				& 1 & RP3     		& Cosmic ray detector \\ \hline
\end{tabular}
\end{table}

\subsubsection{Electronics Design}
Since much of the space inside of our payload will be taken up by the pumps and clean box from the astrobiology systems we had to design our electronics to be relatively compact.  It was decided that we would only use one RP3 to both interface with MiniPIX and store sensor data from the Arduino.  Also, in order to reduce the space required for the interface between the Arduino and all of the payload's sensors, we will use two layers of proto shields to more effectively utilize vertical space.
The RTC, pressure, humidity, and inertial measurement sensors will be mounted directly on the first shield while the temperature are mounted on to the top most shield. 

\subsubsection{Telemetry}
RESU will handle every subsytem during the mission.  It will handle all telemetry to and from the HASP systems.  Downlinked packets will provide timestamped readings from all the payload's sensors which help us in analyzing the current state of the payload.  RESU will also receive four uplink commands: heaters on/off and pumps on/off.  These commands will be sent at our discretion which will allow us to manage our collection systems at any given point during the flight based on environmental and component temperatures. 
