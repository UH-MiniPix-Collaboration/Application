\subsection{RESU Design}
\label{sec:RESU-Design}

\subsubsection{Overview}

For this mission, we will again use RESU (Real-Time Environmental Sensing Unit), our flight computer to manage flight operations and send and receive commands from the ground. RESU is composed of two components: a Raspberry Pi 3 (RP3) and an Arduino Mega (Arduino) which interfaced over serial USB. The RESUs primary purpose during the flight will be to monitor environmental conditions and control the astrobiology systems. It will monitor temperature of the various subsystems and the humidity and pressure of the environment throughout the flight. All recorded data will continuously be written to an SD card mounted on a shield on top of the Arduino. It will also accept discrete commands from the HASP systems to turn the astrobiology collection system on and off.

   The RP3 and Arduino both are hobby electronic computers, but the RP3 is geared towards recording and processing while the Arduino is more efficient at sensor integration.  The Arduino's primary purpose is to collect data from various sensors that monitor environmental conditions. It also can handle telemetry, periodically downlinking packets and accepting uplink commands received from the ground control. The RP3 records data from the Arduino's serial port and saves data frames every six seconds from the MiniPIX.  Table~\ref{tab:Sensors} lists the various sensors that will be utilized during flight.

\subsubsection{The Sensors}

Our payload will utilize eight thermistors to measure temperature at various points in our payload. The decision to use thermistors was based primarily on the performance of the analog temperature sensors during our 2017 flight, during which several of those sensors failed. Thermistors are able to accurately measure temperature in the range \SIrange{-55}{125}{\celsius} and should therefore be adequate for the conditions in the stratosphere. Pressure will be recorded from two identical digital pressure sensors in order to ensure accuracy and should be able to record accurately in the range \SIrange{0}{14000}{\milli\bara}. Finally, humidity will be measured from a basic analog humidity sensor. All sensor data will be UTC timestamped via the onboard Real Time Clock and recorded to the SD card.

\begin{table}[h!]
\centering
\caption{Table of sensors that compose RESU}
\label{tab:Sensors}
\bigskip
\begin{tabular}{|c|c|c|c|}
\hline
\multicolumn{1}{|c|}{\bfseries Sensor} & {\bfseries Quantity} & {\bfseries Platform} & {\bfseries Purpose} \\
\hline
    Temperature Sensors         	& 6 & Arduino  		& Record temperature measurements  \\ \hline
    Pressure        				& 2 & Arduino 		& Record pressure measurements \\ \hline
    Inertial Measurement Unit       & 1 & RP3    		& Record IMU Data in 9 degrees of freedom \\ \hline    
    Real Time Clock 				& 2 & Arduino/RP3 	& Record temperature compensated timestamps in CT \\\hline
    Humidity        				& 1 & Arduino 		& Record atmospheric humidity levels \\ \hline
    MiniPIX         				& 1 & RP3     		& Cosmic ray detector \\ \hline
\end{tabular}
\end{table}

\subsubsection{Power Supply}

In order to stay within the power constraints, a robust power supply will need to  handle all the components of the payload.  The power supply we will be using is the PPM-DC-ATX-P by WinSystems INC.  It offers the desired number of \SI{+5}{\volt} and \SI{+12}{\volt} outputs needed to power the payload's electronics.  This power supply could effectively take \SI{+30}{\volt} and step it down to two \SI{+12}{\volt} and two \SI{+5 }{\volt} outputs.  One of the \SI{+12}{\volt} outputs goes to the Arduino since it can step down to the appropriate voltages internally while the other goes to a PWM motor for the solenoid.  One of the \SI{+5 }{\volt} outputs powers two analog sensors that will be sent to HASP through the EDAC connection (more on that in the next sections).  The remaining \SI{+5 }{\volt} output is converted to a USB power cable for the RP3.  The power supply also has four ground outputs that will be used by each respective component.  \cmt{Andrew R}{May be good to say that this is the same supply model that we used for the first flight, without issue.}


\subsubsection{Space Constraints}
Since much of the space inside of our payload will be taken up by the pumps and clean box from the astrobiology systems, we need to design our electronics to be relatively compact.  It is decided that we will again only use one RP3 to both interface with MiniPIX and store sensor data from the Arduino.  Also, in order to reduce the space required for the interface between the Arduino and all of the payload's sensors, we will use two layers of proto-shields to more effectively utilize vertical space. The RTC, pressure, humidity, and inertial measurement sensors will be mounted directly on the first shield while the temperature are mounted on to the top most shield. 

