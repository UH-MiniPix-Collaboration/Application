\subsection{Radiation}
\label{sec: Radiation Background}

The study of the biological effects of cosmic radiation in space or near space environments is important for human space travel. Long term space travel necessarily requires humans and other biological specimens to be exposed to high levels of radiation for extended periods of time, so understanding the amount of radiation exposure experienced from galactic cosmic rays (GCRs) has important applications for flights to Mars and beyond. Also, the understanding of cosmic rays specifically in Earth's atmosphere has important applications to commercial airplane flights as they generally operate at an altitude with far higher levels of radiation exposure than at the surface of Earth. 

Our goals for the radiation portion of our payload are two fold, to measure radiation levels at various layers in the atmosphere to determine its possible effects on microorganisms, and to use a MiniPIX particle detector to gather data from GCRs in the atmosphere. Using the results from the previous iteration of SORA~\cite{SORA} as a baseline, we know the capabilites of the MiniPIX, and we will make several design changes to improve the efficiency of our system and to produce more in-depth data analysis. We ultimately seek to further previous research utilizing MediPIX/TimePIX devices for measuring GCRs on stratospheric balloon flights~\cite{bexus}. 



%
%	The biological effectiveness of
%Galactic Cosmic Rays (GCR) in space has been studied on
%microorganisms through the use of computer modeled radiobiological
%systems using Monte Carlo simulations. With an increase in intensity
%of GCR's, the number and magnitude of damage in cells increase 5. Such
%damage occuring from GCRs can effect the biological system.
%
%In the atmosphere ionizing radiation does not always increase with altitude. As
%primary GCRs such as high energy protons, alpha particles, and 
%heavy ions collide with atoms in the atmosphere they begin to shatter into secondary
%particles such as neutrons, pions, electrons and muons which causes a peak in 
%ionizing radiation at around \SI{18}{\kilo\meter}. This peak is known 
%as the Regener-Pfotzer Maximum\cite{regener} and shows that with increasing atmospheric density
% ionizing radiation increases until peaking high in the stratosphere and then decreases rapidly as you 
%reach the surface of the earth.



%Expand upon Regener-Pfotzer Maximum. Refer to other papers to validate the range. 
%The intensity of GCR peaks within a range of about
%\SI{18}{\kilo\meter} to about \SI{22}{\kilo\meter} []. This range,where the production of ionizing radiation reaches its
%peak is known as the Regener-Pfotzer Maximum\cite{regener} . The Regener-Pfotzer Maxiumum is unique and is dependent on
%location and time of the year, as it is determined by a number of
%factors, which include but are not limited to the strength of earth's
%electromagnetic field, atmospheric composition (specifically ozone
%content(?){\bf The intensity of the cosmic ray flux and the secondary
%  environment vary inversely with the solar cycle due to the
%  interaction of the earths electromagnetic field. In addition, the
%  sporadic solar events that occur in short busts can increase the
%  primary particle flux periodically (hours to days) can in fact
%  enhance the atmospheric radiation several orders of magnitude in
%  scale.}), the sun's relative position, and maximum solar elevation
%[]. The combination of these affects results in a variability in the
%location of the maximum as well as the existene of this maximum as
%ooposed to an ever-increasing intensity.

%\subsubsection{Solar Radiation}
%The ultraviolet (UV) spectrum is composed of UVC (\SIrange{200}{280}{\nano\meter}) with only \SI{0.5}{\percent} of the entire solar spectrum, 1.5\% of
%UVB (\SIrange{280}{315}{\nano\meter}) and UVA (\SIrange{315}{400}{\nano\meter}), which contributes to \SI{6.3}{\percent}~\cite{uv_irradiance}. UVB and UVC are the main contributors in highly lethal solar radiation to microorganisms\cite{UVonDNA}. DNA is prone to high absorption levels at those wavelengths, often causing inactivation and mutation.  Therefore, understanding the exposure of microbes in to UV radiation is quite important.

